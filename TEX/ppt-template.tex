    \documentclass{beamer}

    \mode<presentation>
    {
      \usetheme{Darmstadt}
      \setbeamercovered{transparent}
    }

    \usepackage[english]{babel}
    \usepackage[latin1]{inputenc}
    \usepackage{times}
    \usepackage[T1]{fontenc}
    \title[Markdown to PDF] % (optional, use only with long paper titles)
    {Markdown to PDF}

    \subtitle
    {}

    \author[Kashyap] % (optional, use only with lots of authors)
    {C.K.Kashyap}

    \institute[IBM] % (optional, but mostly needed)
    {
      International Business Machines\\
      Hyderabad}

%    \date[bragging] % (optional)
%    {22 July, 2009 / bragging}

    \subject{Computer boot up}

    \pgfdeclareimage[height=0.74cm]{company-logo}{IBM}
    \logo{\pgfuseimage{company-logo}}

    %% sita, from page 67 of the beamer userguide
    \setbeamertemplate{navigation symbols}{}

    %% sita, copied from /usr/share/texmf/tex/latex/beamer/themes/outer/beamerouterthemeinfolines.sty
    %% warning, only tested in Darmstadt theme; may not match others
    %% (especially the vertical nav ones like Hannover).  FIXME: figure out how to
    %% disable it when you use themes using vertical navigation
    \defbeamertemplate*{footline}{infolines theme}
    {
      \leavevmode%
      \hbox{%
      \begin{beamercolorbox}[wd=.333333\paperwidth,ht=2.25ex,dp=1ex,center]{author in head/foot}%
        \usebeamerfont{author in head/foot}\insertshortauthor~~(\insertshortinstitute)
      \end{beamercolorbox}%
      \begin{beamercolorbox}[wd=.333333\paperwidth,ht=2.25ex,dp=1ex,center]{title in head/foot}%
        \usebeamerfont{title in head/foot}\insertshorttitle
      \end{beamercolorbox}%
      \begin{beamercolorbox}[wd=.333333\paperwidth,ht=2.25ex,dp=1ex,right]{date in head/foot}%
        \usebeamerfont{date in head/foot}\insertshortdate{}\hspace*{2em}
        \insertframenumber{} / \inserttotalframenumber\hspace*{2ex}
      \end{beamercolorbox}}%
      \vskip0pt%
    }

    % Delete this, if you do not want the table of contents to pop up at
    % the beginning of each subsection:
    \AtBeginSubsection[]
    {
      \begin{frame}<beamer>{Outline}
        \small
        \tableofcontents[currentsection,currentsubsection]
      \end{frame}
    }

    % If you wish to uncover everything in a step-wise fashion, uncomment
    % the following command:

    % \beamerdefaultoverlayspecification{<+->}

    \begin{document}

    \begin{frame}
      \titlepage
    \end{frame}

    %%#section \begin{frame}{Outline}
    %%#section   \tableofcontents
    %%#section   % You might wish to add the option [pausesections]
    %%#section \end{frame}

    % \section{sectionname}
    % \subsection[short]{long}



    \section{introduction}
	    \subsection{Name}
		    \begin{frame}{This is the story of my name}
		    	\begin{itemize}
				\item{Kashyap}
				\pause
				\item{CK}
			\end{itemize}
		    \end{frame}
		    \begin{frame}{This is the rest of the story of my name}
				\item{Kashyap}
				\pause
				\item{CK}
		    \end{frame}

    \subsection{profession}


\section{the problem}

\subsection{office software sucks}

\begin{frame}{all office software sucks}

\begin{itemize}
\item
but presentation software sucks more
\item
it requires too much mousing around
\begin{itemize}
\item
doesn't matter if it is OpenOffice or MS Office
\item
I hate them both equally
\end{itemize}
\item
and I hate mice
\begin{itemize}
\pause
\item
nasty, smelly, things
\pause
\item
never even have a bath, what a life...
\end{itemize}
\end{itemize}



\end{frame}

\begin{frame}{My presentations}

\begin{itemize}
\item
all these years, my presentations have consisted of
\begin{itemize}
\item
plain text on plain white backgrounds
\item
with hardly any pictures
\begin{itemize}
\pause
\item
I start sweating if I have to make a picture or a chart
\pause
\item
even if I'm using OpenOffice
\end{itemize}
\end{itemize}
\end{itemize}



\end{frame}

\begin{frame}{My preferred editor}

\begin{itemize}
\item
for everything and anything under the sun
\pause
\item
is "\href{http://www.vim.org}{vim}"
\pause
\item
even my firefox browser uses
"\href{http://vimperator.mozdev.org/}{vimperator}", an extension which lets me
use vi keystrokes instead of the mouse :-)
\end{itemize}



\end{frame}

\begin{frame}{so the problem is this:}

\begin{itemize}
\pause
\item
I want to make presentations using plain text
\pause
\item
and if they can look prettier and feel slicker, that's a bonus
\end{itemize}



\end{frame}

\begin{frame}{in the interest of time}

(...and your sanity)
\linebreak
\linebreak
I will not bore you with all the stuff that failed...



\end{frame}


\section{the pieces start to fall in}

\subsection{text to HTML}

\begin{frame}{simple HTML made even simpler}

\begin{itemize}
\item
I'd been using Markdown for a year or so now
\item
Markdown is one of the seventeen thousand or so markup languages in the
world
\begin{itemize}
\item
very simple text to HTML conversion
\item
indentation based for easy lists
\item
italics is like \texttt{*italics*}
\item
bold is like \texttt{**bold**}
\begin{itemize}
\item
...and so on; more details
\href{http://daringfireball.net/projects/markdown/}{here}
\end{itemize}
\end{itemize}
\end{itemize}



\end{frame}

\subsection{LaTeX to PDF}

\begin{frame}{detour: LaTeX}

\begin{itemize}
\item
in the beginning, Don Knuth created TeX
\begin{itemize}
\item
(yes, I know it's blasphemy to not format that correctly)
\end{itemize}
\item
then Leslie Lamport created \href{http://www.latex-project.org/}{LaTeX}
\begin{itemize}
\item
most popular and powerful text processing language in academia
\end{itemize}
\item
then Till Tantau came up with
\href{http://latex-beamer.sourceforge.net/}{Beamer}
\end{itemize}



\end{frame}

\begin{frame}{beamer kicks ass}

\begin{itemize}
\item
it produces PDF
\pause
\item
some fantastic PDF actually
\pause
\item
look at the slide navigation on this one and the other two
\pause
\item
try clicking around to go to other parts of the PDF
\pause
\item
I defy anyone to come up with this kind of navigation in MS or
OpenOffice!
\begin{itemize}
\item
I'm only showing three themes; there are many more
\item
and you can make your own (in fact all these have a subtle mod that my
\texttt{mdbeamer} produces)
\end{itemize}
\end{itemize}



\end{frame}

\begin{frame}{so all I need is...}

...some way to convert HTML to LaTeX-beamer syntax



\end{frame}


\section{the last piece}

\subsection{HTML to LaTeX}

\begin{frame}{is "mdbeamer.pl"}

\begin{itemize}
\item
a 150-line perl program I wrote to convert HTML to beamer
\item
very simple, but handles all the markups I care about
\item
some parts of it feel a little kludgy because of the HTML in between
\end{itemize}



\end{frame}


\section{bonus: images}

\subsection{aka: text to graphics}

\begin{frame}[fragile]{detour: graphviz}

\begin{itemize}
\item
I'd recently discovered \href{http://www.graphviz.org/}{graphviz}
\item
excellent for drawing simple diagrams
\item
for example, this code

\begin{verbatim}
digraph {
    node[fontsize=24]
    a -> b -> c -> d
    b -> p -> q -> x
    p -> y
}

\end{verbatim}

\end{itemize}



\end{frame}

\begin{frame}{detour: graphviz}

\begin{itemize}
\item
produces this:
\end{itemize}




\end{frame}

\begin{frame}[fragile]{and something like this...}

\tiny

\begin{verbatim}
digraph G {
    subgraph clusterCS {
        label="Chennai server\n\ \ \ "

        cs2 [shape=box, label="bare repo\non server", style=filled, fillcolor=green]
        node [style=invis]
        edge [style=invis]
        cs1 -> cs2
    }
    subgraph clusterCL {
        label="Commits on\nChennai Lead PC"

        node [shape=box, style=rounded, style=filled, fillcolor=lightblue]
        c1 [label = "Commit #1\n.gitignore"]
        c2 [label = "Commit #2\nSource\nFiles"]
        c1 -> c2
    }
    cs2 -> c2 [lhead=clusterCL, ltail=clusterCS, label = "push", dir=back, color=red, constraint = false]
}

\end{verbatim}




\end{frame}

\begin{frame}{produces this}




\end{frame}


\section{the end}

\begin{frame}{so really the last piece}

\begin{itemize}
\item
well, I'm taking the HTML and converting it to LaTeX anyway
\item
so, devise a simple syntax to embed graphviz code directly into the text
\item
and make my \texttt{mdbeamer} program
\begin{itemize}
\item
extract that code
\item
call graphviz
\item
produce the image
\end{itemize}
\end{itemize}

And.... we're done.  I can do pretty much everything in text now!  And
\textbf{\textit{everything}} stays in one simple text file!

\end{frame}


\end{document}
