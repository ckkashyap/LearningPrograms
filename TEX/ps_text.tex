\documentclass {book}

\usepackage{pstricks}
\usepackage{pst-plot}
\usepackage{pst-grad}
\usepackage{pst-text}

\begin{document}

\newrgbcolor{myorange}{0.9 0.6 0.2}
\newrgbcolor{mylightorange}{0.95 0.8 0.2}

\color{blue}
Thus we find that $x+y=3$ and
using this together with
\psovalbox[linecolor=red]%
          {$xˆ2+yˆ2=3$}
found earlier, we see that
$x=2$ and $y=1$

\vspace{1cm}

Thus we find that $x+y=3$ and
using this together with
\psovalbox[linecolor=red,%
           boxsep=false]%
          {$xˆ2+yˆ2=3$}
found earlier, we see that
$x=2$ and $y=1$

\vspace{2cm}

\begin{pspicture}(-3,-1)(3,5)
%  \colaxes[labels=none](0,0)(-3,-1)(3,5)
  \psset{linecolor=blue}
  \psplot{-2.2}{2.2}{4 x 2 exp sub}
  \psset{unit=1.15cm,linestyle=none}
  \pstextpath[l]{%
    \psplot{-2.1}{2.1}{4 x 2 exp sub}}{%
    \color{red}\textit{increasing}}
  \pstextpath[r]{%
    \psplot{-2.1}{2.1}{4 x 2 exp sub}}{%
    \color{red}\textit{decreasing}}
  \psset{unit=1.07cm}
  \pstextpath[c]{%
    \psplot{-2.1}{2.1}{4 x 2 exp sub}}{%
    \color{red}\textit{turning}}
\end{pspicture}

\vspace{1cm}

\newcommand{\firstpara}{%
  \scriptsize
  \LaTeX\ has only limited drawing capabilities, while
  PostScript is a page description language which has a rich set of
  drawing commands; and there are programs (such as \textsf{dvips})
  which translate the \texttt{dvi} output to PostScript. So, the
  natural question is whether one can include PostScript code in a
  \TeX\ source file itself for programs such as \textsf{dvips} to
  process after the \TeX\ compilation? This is the idea behind the
  \textsf{PSTricks} package of Timothy van Zandt. The beauty of it is
  one need not know PostScript to use it---the necessary PostScript
  code can be generated by \TeX\ macros defined in the package}

\DeclareFixedFont{\bigsf}{T1}{phv}{b}{n}{1.75cm}
\begin{pspicture}(0,-0.5)(8,3)
  \rput[bl](0,0){% 
    \begin{minipage}{8cm}
      \color{blue}
      \firstpara
    \end{minipage}}
  \pscharclip[linestyle=none,%   
              fillstyle=solid,%
              fillcolor=mylightorange]%
             {\rput[bl](0.25,0.15){% 
                 \bigsf PSTricks}}
      \rput[bl](0,0){%
        \begin{minipage}{8cm}
          \color{blue}
          \firstpara
        \end{minipage}}
  \endpscharclip
\end{pspicture}


\end{document}
