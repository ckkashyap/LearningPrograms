\documentclass{article}

\usepackage{tikz}
\usetikzlibrary{arrows,decorations.pathmorphing,backgrounds,positioning,fit,calc,matrix,chains}


\begin{document}
\title{{\emph Hello World} bootup program for the x86 PC}
\author{C.K.Kashyap}
\maketitle

\begin{abstract}
Curiosity of how computers worked led me to explore the boot up sequence of x86
machines in details. I am sure there would be many more folks out there who are
equally curious about it. I gathered information on this from various places.
This article contains the essence of it all. It's meant for someone with
programming knowledge and a lot of curiousity to be able to to write up a
{\it Hello World} boot up program for an x86 PC.
\end{abstract}

\section*{Boot up process overview} When you switch on a computer (here on,
		computer would imply an x86 machine) among other things, the
processor gets power supply and it starts up.  Even the most powerful x86
processor today starts up in a humble 8086 mode. It's {\emph Instruction
	Pointer (IP)} register is set to some hard coded value that points to
	some address in {\emph Read Only Memory (ROM)} that contains code to do
{\emph Power On Self Test (POST)}. It's the POST code that makes the keyboard
LED's blink at startup.\\ After POST, the first sector is read from the boot
device and written to a specific location on {\emph Random Access Memory (RAM)}
-- at physical address {\it0x7C00} -- and jumps to it. As you can imagine, our
task is to get our ``Hello World'' in the first sector of the boot device.\\

\section*{The {\emph Hello World} program}

\texttt{[bits 16]}\break
\texttt{[org 0x7c00] ; Be aware that the code is going to be loaded at 0x7c00}\break
\texttt{}\break
\texttt{mov ax,cs}\break
\texttt{mov ds,ax}\break
\texttt{mov ax,0xb800}\break
\texttt{mov es,ax}\break
\texttt{}\break
\texttt{mov cx,[ds:len]}\break
\texttt{mov si,hello}\break
\texttt{mov di,0}\break
\texttt{l1:}\break
\texttt{mov ah,[ds:si]}\break
\texttt{mov [es:di],ah}\break
\texttt{inc si}\break
\texttt{inc di}\break
\texttt{mov byte [es:di],0x4f}\break
\texttt{inc di}\break
\texttt{loop l1}\break
\texttt{}\break
\texttt{hello:}\break
\texttt{	db "HELLO WORLD"  }\break
\texttt{len: 	dw \$-hello}\break
\texttt{}\break
\texttt{jmp \$}\break
\texttt{}\break
\texttt{times (510-(\$-\$\$)) db 0}\break
\texttt{dw 0xaa55 ; Indicating the sector is a boot sector}\break
\texttt{}\break
\texttt{times ((512*18*80*2)-(\$-\$\$)) db 0}\break

\section*{Compiling the code} Code in the previous section can be compiled
using nasm. The resulting file is a bootable floppy image. Assuming that you saved the code in a filed called {\emph boot.nasm}, you can compile the same as follows --\\
	      \\
 {\texttt nasm boot.nasm -o floppy.img}
	      \\

You could test it out by using qemu or any other vitualization software. Using qemu, you could do this -- \\
	    \\
{\texttt qemu -hda floppy.img}


\end{document}
