\documentclass{article}

\usepackage{tikz}

\usetikzlibrary{arrows,decorations.pathmorphing,backgrounds,positioning,fit,calc,matrix,chains}


\begin{document}
\title{{\emph Hello World} bootup program for the x86 PC}
\author{C.K.Kashyap}
\maketitle

\begin{abstract}
Curiosity of how computers worked led me to explore the boot up sequence of x86
machines in details. I am sure there would be many more folks out there who are
equally curious about it. I gathered information on this from various places.
This article contains the essence of it all. It's meant for someone with
programming knowledge and a lot of curiousity to be able to to write up a
{\it Hello World} boot up program for an x86 PC.
\end{abstract}
\section*{Boot up process overview}
When you switch on a computer (here on, computer would imply an x86 machine) a number of things happen.
\end{document}
