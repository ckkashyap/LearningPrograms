\documentclass{article}

\usepackage{tikz}
\usetikzlibrary{arrows,decorations.pathmorphing,backgrounds,positioning,fit,calc,matrix,chains}


\begin{document}
\title{{\emph Hello World} bootup program for the x86 PC}
\author{C.K.Kashyap}
\maketitle

\begin{abstract}
Curiosity of how computers worked led me to explore the boot up sequence of x86
machines in details. I am sure there would be many more folks out there who are
equally curious about it. I gathered information on this from various places.
This article contains the essence of it all. It's meant for someone with
programming knowledge and a lot of curiousity to be able to to write up a
{\it Hello World} boot up program for an x86 PC.
\end{abstract}

\section*{Boot up process overview} When you switch on a computer (here on,
		computer would imply an x86 machine) among other things, the
processor gets power supply and it starts up.  Even the most powerful x86
processor today starts up in a humble 8086 mode. It's {\emph Instruction
	Pointer (IP)} register is set to some hard coded value that points to
	some address in {\emph Read Only Memory (ROM)} that contains code to do {\emph Power On Self Test (POST)}. It's the POST code that makes the keyboard LED's blink at startup.\\
		After POST, the first sector is read from the boot device and written to a specific location on {\emph Random Access Memory (RAM)} - at physical address {\it0x7C00}



\end{document}
