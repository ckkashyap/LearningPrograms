\documentclass [a4paper,11pt] {article}

\usepackage{color}

\pagestyle {plain}

\begin{document}

\section*{Contact details}
\begin{description}
\item[Name:] C.K.Kashyap
\item[Email:] ckkashyap@gmail.com
\item[Mobile:] +919448466112
\end{description}

\section*{Summary}

I have over 10 years of industrial experience in developing software ranging
from device drivers to web applications. My experience includes 3 years of
people management at IBM India Software Labs and driving multiple releases via
a geographically distributed cross functional team. I graduated from BITS
Pilani in 2000 with MMS - Master of Management Studies. MMS used to be an
integrated masters degree which was equivalent of BE + MBA. While at BITS, I
was part of the robotics lab and I did a semester long internship at Center for
Artificial Intelligence and Robotics, Bangalore.

Computer programming has been my hobby since school days. I started with BASIC
programming in 1990. I was fascinated with DOS games and in my quest to write
fast games, I moved from BASIC to assembly and C. On the way, I got hooked on to
Intel's protected mode and eventually operating system kernel programming. I have a few
projects registered with github (and earlier google projects) ;  some of the
interesting ones are, a port of minix3 microkernel so that it builds on Linux
and a Ruby based VNC automation engine. More recently, {\it Functional Programming}
has caught my fancy and I have been exploring {\it Haskell} for over two years now. I believe that functional programming is one of the ways of harnessing the multicore era. As a learning exercise, I'm working on a VNC based platform independent interactive graphics rendering library using Haskell ; https://github.com/ckkashyap/Chitra

Professionally, I started my career with Seranova where I worked on Java based
internet solutions. I switched jobs in the earlier part of my career mainly to
get involved with embedded systems.  In the later part, it was for looking for
larger roles where I could contribute more with my deep understanding of
software and exposure to a wide range of it. I am looking for a job where I can
use my computer and management skills to solve larger problems; help people with leveraging
various kinds of automations. In my earlier jobs, I've taken up doing tutorial classes a
number of times and have received very positive feedback ; I've also enjoyed
such sessions a lot. It'll be great if such activities become part of my job
portfolio. 

\section*{Highlights}
\begin{itemize}

\item Developed a Virtualization infrastructure using qemu/kvm that allowed folks
in my team to quickly generate and work with virtual machines

\item Filed a patent around image processing based computer UI automation, 
with a POC implementation. I also did a rudimentary implementation using VNC and
OpenCV

\item Very interested in operating system development. Implemented multiple
monolithic x86 kernels as hobby projects. Currently fascinated by microkernels.

\item Have been pursuing functional programming – Haskell in particular.

\item Managed product development teams through multiple major releases and
fix-packs  

\item Driven beta programs

\item Experienced in managing and co-ordinating globally distributed
cross-functional teams

\item Deep understanding of software development

\item Deployed innovative virtualization solutions with ESX and qemu for
productivity enhancement

\item Have over 10 years of experience in software development with 3 years in
people management.

\item Graduated with integrated Masters degree in Management studies from {\bf BITS
Pilani} in the year 2000
\end{itemize}

\section*{Areas of technical expertise}
\begin{itemize}
	\item C programming
	\item x86 Assembly programming
	\item Perl programming
	\item Ruby programming
	\item Java programming
	\item Have explored the linux kernel between 2000 and 2005
	\item I’ve been exploring a variety of open source unix type kernels including tabos, minix, newos, HelenOS
	\item Computer Programming - over 19 years (started with BASIC in the year 1990 in my 7th grade)
	\item Worked on building Linux 0.01 using newer GCC and nasm and booting it using GRUB
		\begin{itemize}
			\item \color{blue}{\begin{verbatim} http://kashyap-1978.tripod.com/Linux_0.01.html \end{verbatim}}
		\end{itemize}
	\item Worked on getting Minix3 to build on Linux platform and boot using GRUB
		\begin{itemize}
			\item \color{blue}{http://github.com/ckkashyap/newminix}
		\end{itemize}
	\item Developed automation of VNC using Ruby
		\begin{itemize}
			\item \color{blue}{http://github.com/ckkashyap/robovnc}
		\end{itemize}
	\item Working on a platform independent graphics library using VNC
		\begin{itemize}
			\item \color{blue}{http://github.com/ckkashyap/Chitra}
		\end{itemize}
\end{itemize}


\section*{Professional experience}


\subsection*{Yahoo!}
\begin{description}
\item{Title:} Principal Engineer - performance tuning
\item{Duration:} May 2010 - 
\item{Description:} I've joined the performance engineering group which is a
horizontal group that consults with development groups for performance
enhancements. 
\end{description}


\subsection*{IBM - India Software Labs}
\begin{description}
\item{Title:} Engineering Manager
\item{Duration:} July 2007 - May 2010
\item{Description:} In this role, I've done the following:
	    \begin{description}
	    \item{\bf People Management} Did performance management of engineering team composed of junior engineers, senior engineers, leads and testers.
	    \item{\bf Release Management} Worked on multiple major and fixpack releases. Represented the development team in the cross functional forums.
	    \item{\bf Beta Management} Worked closely with the Beta co-ordinator. Did customer presentations on the upcoming features.
	    \item{\bf Automation} I've implemented various automations to improve my productivity and team's productivity.
	    	\begin{itemize}
		\item{\bf Virtualization Solution using Qemu}\\I implemented a bunch of scripts that allowed creation of various virtual machine images in seconds.
		\item{\bf Virtualization Solution using ESX}\\We had trouble
		getting unique IP addresses for each VM, so I created a linux
		VM and made it a router and the DHCP server. This way all the
		VM's could get unique IP address automatically and also talk to
		outside machines. 
		\item{\bf Quick reporting from RTC data}\\
		When I found that creating required reports from an RTC server
		that was located remotely was eating away too much time, I
		implemented a {\it cron} job that fetched data regularly and created
		local HTML reports that I could view anytime instantaneously. I
		used CURL to fetch data from RTC via its REST API.
		\item{\bf Quick reporting from ClearQuest data}\\
			ClearQuest has API that can be accessed via Perl. I
			used the API via Perl to generate reports via {\it cron} jobs
			because the server very slow to access.
		\end{itemize}
	    \end{description}

\end{description}

\subsection*{IBM - India Software Labs}
\begin{description}
\item{Title:} Staff Software Engineer
\item{Duration:} December 2005 to July 2007
\item{Description:}I started out by leading a two member team that worked on Linux. Once the Linux issues were stabilized, I was moved to a larger role of leading the core modules of the product. This involved, hands on work and also mentoring a group of four members. This product is based on a couple of IDE’s – one which is eclipse based and the other which is Visual studio based. There are some native components that are in C++.
\end{description}


\subsection*{Microsoft}
\begin{description}
\item{Title:} Software Design Engineer
\item{Duration:} March 2005 to December 2005
\item{Description:}Work in the project involves maintenance of Microsoft Virtual Server and Virtual PC. Was involved with 2 fixpack releases of Virtual PC and the release of Virtual Server 2005.
\end{description}

\subsection*{Sun Microsystems}
\begin{description}
\item{Title:} Staff Software Engineer
\item{Duration:} September 2004 to March 2005
\item{Description:}As a part of this team I was responsible to fix bugs reported on the Sun's Webserver. Also, I was in charge of setting automated GAT(General Acceptance Test). Automated GAT involved writing Perl script to continuously check out the source, do a build and run the test suite. Since the repository was in a remote location, I had implemented a kind of "double buffer" to speed up the process.
During the course of work here, I had opportunity to understand PKI
\end{description}


\subsection*{Insilica Semiconductors}
\begin{description}
\item{Title:} Senior Software Engineer
\item{Duration:} October 2003 to September 2004
\item{Description:}The goal of this project was to implement the 802.11 protocol stack. During the development of the stack, the wireless hardware was still being fabricated. Therefore, ethernet was used to emulate the physical layer.  Also, an extensive test-suite was designed and developed in perl to test the stack from a central controller.
\end{description}


\subsection*{Virtusa}
\begin{description}
\item{Title:} Software Engineer
\item{Duration:} March 2003 to October 2003
\item{Description:}This project involved writing an expression compiler for the Pega system's rules engine. ANTLR (lex/yacc equivalent in the java world) was used for this purpose.  Developing the front end of this project involved extensive use of java-script to dynamically make HTTP connections to the server and parsing the retrieved XMLs.
\end{description}


\subsection*{Virtusa}
\begin{description}
\item{Title:} Software Engineer
\item{Duration:} March 2002 to February 2003
\item{Description:}The goal of the EMC EDM Linux Port project is to port EDM (EMC's Data Manager)from Solaris to Linux. EDM contains 2.5 million lines of C and C++ programs. The UI part of EDM is in Java.  The project was carried out in multiple phases. The first phase involved replacing the Solaris specific calls in the application source base with the POSIX equivalent. The second phase involved compiling the whole thing on Linux. The third phase involved executing the application and debugging on both the platforms.
\end{description}

\subsection*{Seranova}
\begin{description}
\item{Title:} Software Engineer
\item{Duration:} November 2001 to March 2002
\item{Description:}This project involved doing night shifts and waiting for client calls and resolving issues with SIF
\end{description}

\subsection*{Seranova}
\begin{description}
\item{Title:} Software Engineer
\item{Duration:} July 2000 to October 2001
\item{Description:}Standard Implementation Framework (SIF) is the web content-management utility developed for a fortune 500 American client. This utility was developed using Vignette's Story-Server. Using the tool a person with no knowledge of HTML could create and update web-content of the corporate web site.  The tool allowed creation of page layouts on the fly. This was achieved using a JAVA applet. All the information about the layout and web contents were stored using XML. I was a developer in one of the modules. My training period was till December 2000.
\end{description}

\end{document}
